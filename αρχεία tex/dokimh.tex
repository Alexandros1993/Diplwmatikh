\documentclass[11pt,a4paper]{article}
\usepackage{xltxtra}

\usepackage{xgreek}

\setmainfont[Mapping=tex-text]{GFS Didot}

\usepackage{amsfonts}

\begin{document}

Όταν θέλουμε να γράψουμε μαθηματικές σχέσεις πρέπει να τις βάλω ανάμεσα σε δολάρια. Ο αριθμός $\sqrt[12]{2}$ παίζει σημαντικό ρόλο στη μουσική.


Μπορώ εύκολα να βάζω δείκτες και εκθέτες $x^y_z$. Παιχνιδάκι είναι και τα κλάσματα είτε έτσι $1/2^2$ είτε είτε έτσι $\frac{1}{2^2}$


Στα προηγούμενα οι σχέσεις μου είναι μέρος του κειμένου. Πολλές φορές όμως θέλω να βρίσκονται στο κέντρο σε ξεχωριστή γραμμή.

$$\lim_{n\to\infty}{a_{k_n}}$$


Τότε τα μαθηματικά πρέπει να μπαίνουν ανάμεσα σε δύο ζεύγη δολαρίων.


Η συνάρτηση $f:\mathbb{R}\rightarrow\mathbb{N}$ με $f(x)=[|x|+1]!$

\end{document}